\documentclass{beamer}
%\documentclass[handout,xcolor=pdftex,dvipsnames,table]{beamer}

%\usepackage[latin1]{inputenc}
%\documentclass[ucs]{beamer}%para sistemas com ucs
\usepackage[utf8]{inputenc}
\usepackage{verbatim}
\usepackage{graphicx}
\usepackage{lmodern}
\usepackage{textcomp}

\usepackage{tikz}   %TikZ is required for this to work.  Make sure this exists before the next line
\usepackage{pgf}

 \usepackage[scaled]{helvet}
 \renewcommand*\familydefault{\sfdefault} %% Only if the base font of the document is to be sans serif
 \usepackage[T1]{fontenc}

%\usepackage[utf8x]{inputenc}%idem
\usepackage[brazil]{babel}
\usepackage{verbatim}
\usetheme{Unicamp}
%\usecolortheme{seagull}
%\usecolortheme{beaver}
%\usecolortheme{seahorse}
\usefonttheme[onlysmall]{structurebold}

\title[Defesa de Doutorado]{Diversidade e estrutura genética de populações de
\textit{Poecilia vivipara} (Cyprinodontiformes: Poeciliidae) do norte do estado do Rio de Janeiro}
\author {Carlos Henrique Tonhatti}
\institute[Unicamp]{Universidade Estadual de Campinas}
\date{11 de dezembro de 2017}



 \AtBeginSection[]
 {
\begin{frame}
\frametitle{Sumário}
\tableofcontents[currentsection]
\end{frame}
}

\AtBeginSubsection[]
{
   \begin{frame}
       \frametitle{Sumário}

      \tableofcontents[currentsection,currentsubsection]
        \setcounter{tocdepth}{2}
    \end{frame}
 }


\begin{document}
%para criar a pagina de rosto
 \frame{\titlepage} %inclui a front page 

% %==================================================slide

% cria o sumario
\begin{frame}
 \frametitle{Sumário}
 \tableofcontents[pausesections]
  \setcounter{tocdepth}{2}% profundidade do sumario 
\end{frame}
%-------------------------------------------------------
%==================================================slide

%criando um slide



\section{Diversidade em sistemas biológicos}
\begin{frame}
  \frametitle{Sistemas biológicos possuem variação}
  \begin{center}
    \pgfdeclareimage[width=10cm]{diversidade}{figures/diversidade}
    \pgfuseimage{diversidade}
  \end{center}
  %%> Sistemas biológicos possuem variacoes em diversos níveis. Neste slide esta representado uma parte dessa variacao. Aqui temos a variacao em racas de caes domesticados. Variacao de coloraćao entre especies de sapos da mata atlantica do mesmo genero. Entre espécies de insetos e entre variedades de batatas.  A variacão nao se limita apenas a variacão morfologica mostrada aqui. Há variaćao tambem em outros nivies como molecular e tambem em comportamento.
\end{frame}

\begin{frame}{Principais perguntas}

  \begin{itemize}[<+->]
  \item  Fontes da variação
  \item  Organiza\c{c}ão da variação
  \end{itemize}
  \vspace{20pt}
% \pause
% \begin{block}{Neste trabalho}
%  Como a varia\c{c}ão no nível molecular de popula\c{c}ões de \textit{P. vivipara} está organizada?
% \end{block}

  %%> Ao se trabalhar com sistemas biologicos duas principais perguntas emergem. Como a variaćao surge em um sistema inicialmente uniforme, ou seja, a fonte dessa variacáo e como a variacao é organizada entre diferentes estratos, populacoes, etc. Nesse trabalho eu foco na segunda questão. Como a variacao no nivel molecular de poulacoes de p. vivipara está organizada. 
  
\end{frame}


\begin{frame}{Ambiente conecta entre as populações}
   \centering
  \pgfdeclareimage[width=10cm]{homo}{figures/homo}
    \pgfuseimage{homo}

  \end{frame}

  \begin{frame}
      \frametitle{Ambiente influencia a organiza\c{c}ão da varia\c{c}ão}

  \centering
  \pgfdeclareimage[width=10cm]{barreiras}{figures/fluxoRe}
    \pgfuseimage{barreiras}

   \end{frame}

\begin{frame}
  \frametitle{Ambiente influencia a organiza\c{c}ão da varia\c{c}ão}
\centering
  \centering
  \pgfdeclareimage[width=10cm]{barre}{figures/barre}
    \pgfuseimage{barre}
\end{frame}

% \begin{frame}
%   \frametitle{Ambiente influencia a organiza\c{c}ão da varia\c{c}ão}
% \centering
%   \centering
%   \pgfdeclareimage[width=10cm]{ajuda}{figures/ajuda}
%     \pgfuseimage{ajuda}
% \end{frame}

\begin{frame}{Ambiente influência variação}{Varia\c{c}ão molecular em popula\c{c}ões humanas atuais}

   \centering
  \pgfdeclareimage[width=10cm]{haplohum}{figures/haplohum}
    \pgfuseimage{haplohum}
  

\end{frame}


\section{Sistema de estudo}
\begin{frame}
  \frametitle{Sistema de Estudo}{Popula\c{c}ões de \textit{Poecilia vivipara } no norte do estado do Rio de Janeiro}
  \begin{columns}
    

    \begin{column}{4cm}
      \centering

      
       \pgfdeclareimage[width=3.8cm]{poe}{figures/3poe}
      \pgfuseimage{poe}<1->\\

    \end{column}
    \begin{column}{8cm}
     \pause
       \centering
       \pgfdeclareimage[width=8cm]{mapa}{figures/RPSintro}
      \pgfuseimage{mapa}\\
     \end{column}

   \end{columns}

\end{frame}





\begin{frame}
  \frametitle{ Poecilideos  apresentam  grande variação}
\centering


  \centering
  \pgfdeclareimage[width=12cm]{viviparas}{figures/viviparas}
  \pgfuseimage<2>{viviparas}
  

    \pgfdeclareimage[width=12cm]{poecilias}{figures/poecilias}
    \pgfuseimage<1>{poecilias}


\end{frame}




\begin{frame}
  \frametitle{Formação do delta do Rio Paraíba do Sul}

  \begin{columns}
    

    \begin{column}{4cm}
      \centering
       \pgfdeclareimage[height=5cm]{mapa1}{figures/mapa1}
      \pgfuseimage{mapa1}<1->\\
Há 5100 anos
    \end{column}
    \begin{column}{4cm}
     \pause
       \centering
       \pgfdeclareimage[height=5cm]{mapa2}{figures/mapa2}
      \pgfuseimage{mapa2}\\
     Há 4000 anos
    \end{column}

    \begin{column}{4cm}
\pause
      \centering
       \pgfdeclareimage[height=5cm]{mapa3}{figures/mapa3}
      \pgfuseimage{mapa3}\\
 Atual
    \end{column}
  \end{columns}  


\end{frame}



\section{Proposta de trabalho}
\begin{frame}
  \frametitle{Objetivos}

  \begin{itemize}[<+->]
\item Descrever da variação genética em \textit{Poecilia vivipara}
\item  Avaliar a  variação dentro  e entre populações (\textbf{estrutura genética})
\end{itemize}
\pause
\begin{block}{Marcadores utilizados}
    \begin{itemize}
\item Região de Controle da Replica\c{c}ão Mitocondrial (DNA mitocondrial)
\item Microssatélites (DNA nuclear)
\end{itemize}
 
\end{block}
%%> Neste trabalho o foco foi a descricao da variacao genetica em P. vivipara e a organizaćao intra e entre populacoes. E com isso tentar compreender a papel do ambiente como forca que organiza a diversidade.
\end{frame}

\begin{frame}{Coleta das amostras populacionais}
\centering

    \pgfdeclareimage[height=8cm]{Rplot05}{figures/Rplot05}
    \pgfuseimage{Rplot05}
%%> Para poder avaliar o impacto da formacao do delta na organizaćao nós amostramos 14 populacoes dispostas como mostrado na figura. Sendo 3 populaćoes externas ao delta do RPS. 5 populacoes da área de formacao com  influencia fluvial e 6 populacoes da area de formacao com influencia marinha. De cada ponto foram coletados 30 individuos que foram eutanaziados e tiveram os tecidos extraidos.
\end{frame}

\begin{frame}
  \frametitle{Biblioteca de microssatélite}
  \centering
     \pgfdeclareimage[height=8cm]{biblioteca}{figures/biblioteca}
    \pgfuseimage{biblioteca}
%%> Para este trabalho fora necessário desenvolver uma biblioteca enriquecida em microssatelites. A técnica utilizada consiste na fragmentaćao do DNA genomico, ligacáo do adaptadores, hibridizacao com sondas , selecao das regioes com motivos de microssatelites, seguida de eluicao e clonagem em vetores de sequenciamento.
\end{frame}

\begin{frame}{Metodologia}
\centering

    \pgfdeclareimage[height=8cm]{fluxo}{figures/fluxo}
    \pgfuseimage{fluxo}

%%> O procedimento feito no laboratorio esta resumido neste fluxograma. O DnA é extraido do tecido e passa por reacoes de amplificacao com primers especificos. No caso da regiao de controle o produto de PCR foi sequenciado e se tivesse qualidade a sequencia entrava para o banco de dados. Para os microssatelites após a amplificaćao o produto de amplificacao passava pela genotipagem  e se tivesse qualidade minima entrava para o banco de dados. Em ambos os casos se q aulidade do procedimento não fosse adequada os dados daquele individuo voltava para a fase de extracao.    
\end{frame}

\section{Resultados}






\begin{frame}{Variação da Região de Controle}
\begin{table}[ht]
\caption{Índices de diversidade  molecular da região de controle do \textit{Poecilia vivipara} em  14 populações}
 \centering 


   \scalebox{0.8}{
\begin{tabular}{@{\extracolsep{5pt}} lccccc} 

\hline \\[-1.8ex] 
 Popula\c{c}ão & n & S & nh & $h$ & $\pi$ \\ 
\hline \\[-1.8ex] 
 B. do Itabapoana (BI) & $28$ & $7$ & $2$ & $0,071$ & $0,001$ \\ 
Cima (CI) & $30$ & $7$ & $7$ & $0,552$ & $0,001$ \\ 
Campelo (CA & $30$ & $16$ & $8$ & $0,747$ & $0,003$ \\ 
Comércio (CO) & $29$ & $17$ & $5$ & $0,594$ & $0,008$ \\ 
Taí (TA) & $29$ & $18$ & $7$ & $0,670$ & $0,008$ \\ 
Grussaí (GR)& $29$ & $19$ & $8$ & $0,791$ & $0,008$ \\ 
Feia (FE)& $30$ & $19$ & $10$ & $0,825$ & $0,008$ \\ 
A\c{c}ú (AC)  & $28$ & $20$ & $10$ & $0,815$ & $0,009$ \\ 
Buena (BU) & $29$ & $15$ & $5$ & $0,756$ & $0,007$ \\ 
Ururaí  (UR) & $24$ & $19$ & $8$ & $0,757$ & $0,007$ \\ 
Itaocara (IT) & $20$ & $14$ & $4$ & $0,679$ & $0,008$ \\ 
Imburí (IM)& $30$ & $0$ & $1$ & $0$ & $0$ \\ 
Cabiúnas (CB) & $28$ & $3$ & $4$ & $0,206$ & $0,0003$ \\ 
Atafona  (AT) & $22$ & $18$ & $9$ & $0,874$ & $0,005$ \\
 


\hline \\[-1.8ex] 
\end{tabular} }
\label{descritiva}
\end{table}
\end{frame}

\begin{frame}{Variação da Região de Controle}

   \centering
             \pgfdeclareimage[width=11cm]{mapahap}{figures/mapahap}
      \pgfuseimage{mapahap}\\

\end{frame}


\begin{frame}
  \frametitle{Desenvolvimento  da biblioteca de microssatélites}

     \pgfdeclareimage[height=8cm]{classe}{figures/classe}
    \pgfuseimage{classe}
%%> O exito do enriquecimento da biblioteca de microssatelites pode ser visto nesste slide. A maior parte dos microssatelites encontrados foram da classe dinucleotideo. Com maior frequencia nos microssatelites com 6 repeticoes. Foram identificados microssatelites com mais de 30 repeticoes . As frequencias nas outras classes foram bem menores como pode ser visto nestes outros graficos 
  \end{frame}

\begin{frame}
  \frametitle{Desenvolvimento dos locos microssatélites}
  \begin{itemize}[<+->]
  \item 24 locos foram desenvolvidos
  \item 18 locos mostraram polimórficos
  \item 16 locos foram genotipados com sucesso em todas as popula\c{c}ões    
    \end{itemize}
  \end{frame}

\begin{frame}{Varia\c{c}ão nos locos microssatélites}

  \begin{table}
  \begin{center}
  \caption{Caracteriza\c{c}ão dos locos microssatélites nas popula\c{c}ões de  \textit{Poecilia vivipara} amostradas.} 
  \scalebox{0.8}{
\begin{tabular}{@{\extracolsep{5pt}} cccc} 
 \hline
Loco & N\textordmasculine de Alelos & Menor tamanho (pb)  & Maior tamanho (pb)  \\ 
  \hline
PVM01 & $5$ & 186 & 194 \\ 
PVM02 & $6$ & 198 & 234 \\ 
PVM03 & $11$ & 264 & 302 \\ 
PVM04 & $8$ & 202 & 224 \\ 
PVM05 & $30$ & 128 & 220 \\ 
PVM06 & $8$ & 146 & 166 \\ 
PVM07 & $18$ & 176 & 214 \\ 
PVM08 & $8$ & 200 & 222 \\ 
PVM09 & $5$ & 124 & 138 \\ 
PVM10 & $6$ & 200 & 226 \\ 
PVM11 & $38$ & 138 & 298 \\ 
PVM12 & $17$ & 204 & 268 \\ 
PVM13 & $9$ & 194 & 212 \\ 
PVM14 & $4$ & 172 & 184 \\ 
PVM15 & $13$ & 214 & 244 \\ 
PVM16 & $40$ & 204 & 318 \\ 

   \hline
\end{tabular}
 }
\end{center}
\end{table}
  
  \end{frame}

\begin{frame}{Estrutura}{Redes de haplótipos  mitocondriais}
 \centering
       \pgfdeclareimage[width=12cm]{rede}{figures/rede}
      \pgfuseimage{rede}\\ 
\end{frame}




\begin{frame}{Estrutura}{Parti\c{c}ão da Varia\c{c}ão mitocondrial}

  
  \begin{columns}
    


      \begin{column}{7cm}
       \centering
       \pgfdeclareimage[height=7cm]{tm}{figures/tm}
      \pgfuseimage{tm}\\

    \end{column}
    \begin{column}{6cm}
      \centering
       \pgfdeclareimage[height=6cm]{amova1}{figures/amova1}
       \pgfuseimage{amova1}<1>

        \pgfdeclareimage[height=6cm]{amova2}{figures/amova2}
        \pgfuseimage{amova2}<2>

        \pgfdeclareimage[height=6cm]{amova3}{figures/amova3}
        \pgfuseimage{amova3}<3>
        \pgfdeclareimage[width=5cm]{mapahap2}{figures/mapahap}
        \pgfuseimage{mapahap2}<4>
        
    \end{column}

  
  \end{columns}  

  
\end{frame} 

% \begin{frame}{ Estrutura}{\textit{PCA}}

%     \centering
%              \pgfdeclareimage[width=10cm]{PCA}{figures/PCAsem_externo}
%       \pgfuseimage{PCA}\\

  
%     \end{frame}
    
\begin{frame}{Estrutura}{Atribui\c{c}ão dos indivíduos usando locos microssatélites}
  \pgfdeclareimage[width=12cm]{mapaST}{figures/mapaST}
    \pgfuseimage{mapaST}

\end{frame}



\begin{frame}{ Estrutura}{\textit{PCA}}

    \centering
             \pgfdeclareimage[width=10cm]{PCA}{figures/PCAsem_externo}
      \pgfuseimage{PCA}\\

  
    \end{frame}

\begin{frame}
  \frametitle{Conclusões}

  \begin{itemize}[<+->]
\item Há uma grande varia\c{c}ão  genética em \textit{Poecilia vivipara}
\item A varia\c{c}ão está organizada geograficamente
\end{itemize}

 

%%> Neste trabalho o foco foi a descricao da variacao genetica em P. vivipara e a organizaćao intra e entre populacoes. E com isso tentar compreender a papel do ambiente como forca que organiza a diversidade.
\end{frame}

% \begin{frame}{Demais achados}

% \end{frame}

\begin{frame}{Agradecimentos}
  \centering
 \pgfdeclareimage[width=10cm]{agrad}{figures/agrad}
    \pgfuseimage{agrad}

\end{frame}

\section*{Bibliografia}
\begin{frame}{Bibliografia, fontes, etc}
 \centering
 \pgfdeclareimage[width=8cm]{github}{figures/github}
    \pgfuseimage{github}
\end{frame}



\end{document}
  
